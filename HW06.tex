\documentclass{article}
\usepackage{amsmath,amsthm,amsfonts,microtype,amssymb}
\usepackage{enumitem}
\usepackage{parskip}
\usepackage{mathpazo}
\usepackage{tikz-cd}
\usepackage{geometry}
\geometry{top=1in}
\usepackage{float}
\usepackage{graphicx}

\usepackage{hyperref}
\hypersetup{%
  colorlinks=true,
  linkcolor=blue,
  filecolor=blue,
  urlcolor=blue,
  bookmarks=true,
  pdfpagemode=FullScreen,
}


\title{Homework 06 \\ Chain Complexes \& Singular Homology}
\author{Algebraic Topology - Winter 2021}
\date{Due: \textbf{March 11, 2021, 11:59 pm}}

\begin{document}
\pagenumbering{gobble}
\maketitle

\begin{enumerate}
    \item For each of the rings below, compute the homology (up to isomorphism) of the following chain complex over $R$ concentrated in degrees 0 and 1,
    \begin{align*}
        0 \to R^2 \overset{\partial_1}{\longrightarrow} R^2 \to 0,
    \end{align*}
    where $\partial_1 = \begin{bmatrix} 2 & 0 \\ 3 & 1\end{bmatrix}$ in the standard bases.
    \begin{enumerate}
        \item $ R = \mathbb{Q}$,
        \item $ R = \mathbb{Z}/p \mathbb{Z}$ where $p$ is a prime number,
        \item $ R = \mathbb{Z}$.
    \end{enumerate}
    For part (c) express your answers as a (direct sum of) cyclic group(s).
    \item Prove that if $X$ is a point then its singular homology is given by
    \begin{align*}
        H_i(X; \mathbb{Z}) \cong 
        \begin{cases}
        \mathbb{Z} & \mbox{ if } i = 0, \\
        0 & \mbox{otherwise.}
        \end{cases}
    \end{align*}
    
    \item Let $n$ be a non-negative integer. Let $(C_\bullet, d_\bullet)$ be a chain complex over a field $\mathbb{F}$ such that $C_i$ is a finite-dimensional vector space for each $ 0 \le i \le n$ and is 0 for $ i < 0$ or $i > n$. 
    \begin{align*}
        0 \to C_n \overset{d_n}{\longrightarrow} C_{n-1}
        \overset{d_{n-1}}{\longrightarrow} C_{n-2}
        \overset{d_{n-2}}{\longrightarrow} 
        \dots
        \overset{d_{2}}{\longrightarrow} C_1
        \overset{d_{1}}{\longrightarrow} C_0
        \to 0
    \end{align*}
    Show that 
    \begin{align*}
        \sum \limits_{i=0}^n (-1)^i\dim H_i (C_\bullet) =        
        \sum \limits_{i=0}^n (-1)^i\dim C_i.
    \end{align*}
    Thus, either of the two sides can be used to define the Euler characteristic of $(C_\bullet, d_\bullet)$.
    
    % \item
    % Let $X$ be a topological space.
    % Show that if $\{ X_i \}_{i \in I}$ are the path-connected components of $X$ then for all $j \in \mathbb{Z}$
    % \begin{align*}
    %     H_j(X; \mathbb{Z}) \cong \bigoplus \limits_{i \in I} H_j(X_i; \mathbb{Z}).
    % \end{align*}
    % (Be careful with your arguments for this question.)

    
\end{enumerate}
\end{document}
